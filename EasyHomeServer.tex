\documentclass[a4paper,12pt]{report}
\usepackage[a4paper,textwidth=39em,vmargin=25mm]{geometry}
\usepackage{calc,tocloft,datetime2,listings,paralist,fontspec,tabu}
\usepackage{graphicx,eso-pic}
\usepackage[PunctStyle=plain,RubberPunctSkip=false,CJKglue=\hskip 0pt,CJKecglue=\hskip 4pt plus 20pt]{xeCJK}
\usepackage{xeCJKfntef}
\usepackage[hidelinks]{hyperref}

\setmainfont{Nimbus Roman}

\setlength{\parindent}{0pt}
\setlength{\parskip}{8pt}
\linespread{1.1}

\frenchspacing


\title{Easy Home Server}
\author{Neruthes}
\date{Version \today}

\begin{document}
\newgeometry{a4paper,textwidth=39em,bmargin=25mm,tmargin=70mm}

\begin{titlepage}
    \center
    \huge Easy Home Server\par\vskip 25pt
    \normalsize Neruthes\par\vskip 15pt
    \normalsize \today\par\vskip 15pt
    \vfill
    \small
    You may get the source code of this publication at\\
    \href{https://github.com/neruthes/easy-home-server}{https://github.com/neruthes/easy-home-server}\par
    Licensed under \href{https://creativecommons.org/licenses/by-nc-sa/4.0/deed.en\_US}{CC BY-NC-SA 4.0}
\end{titlepage}
\restoregeometry







\chapter*{Preface}

I intend to develop a manual on home server management,
for various purposes including NAS (network-attached storage) and OSS (object storage service).
The setups documented in this manual are derived from my personal setup.
I wish these techniques will empower your home server as well.

When you read this manual, you should at least have entry-level GNU/Linux operation skills,
like installing software packages, managing services, and using CLI.

If you would like to have a word with me,
you may create an issue at the GitHub repo page of this project:
\href{https://github.com/neruthes/easy-home-server}{https://github.com/neruthes/easy-home-server}.
Feedbacks are appreciated.





% \setcounter{tocdepth}{1}
\clearpage
\tableofcontents
\clearpage





\chapter{Generic NAS}



\section{Introduction}

NAS (network-attached storage) is a class of network-based file storing service or a class of hardware specialized for this kind of service.
You may have certain reasons to avoid commercial special-purpose hardware solutions (e.g. Synology, QNAP)
and all-in-one free/open-source solutions (e.g. TrueNAS).
In this chapter, I will discuss how to configure a general-purpose GNU/Linux home server for this purpose.
Personally, I prefer this way because I believe in the KISS philosophy (`keep it simple, stupid').

People who need NAS may include video studios, digital ownership freaks, and science laboratories.
By maintaining a NAS service in your home or studio, you will have a better file-sharing experience.



\section{Space Allocation}

You need to allocate a root directory. Suppose we will use `\texttt{/var/www/nasroot}' for this purpose.

If your root filesystem is not large enough, you may make the directory a mount point (\texttt{man 8 mount})
and mount another partition over there.

If you have data security concerns, you may encrypt the entire external partition or encrypt the directory.

To encrypt a partition, you need LUKS
\footnote{LUKS: See \href{https://wiki.archlinux.org/title/Dm-crypt/Encrypting_a_non-root_file_system}{Arch Linux Wiki} for more details.}
(\texttt{man 8 cryptsetup}).
This approach lacks flexibility in several senses, but I prefer not to explain them here.

To encrypt a directory, you need \href{https://github.com/vgough/encfs}{EncFS}
\footnote{EncFS: A FUSE-based file encryption software developed by Valient Gough.} or something similar.
This approach frees you from allocating a fixed amount of disk space before setting up the service.



\section{Web Service}

You need to install and setup \textbf{Nginx}.

\section{Uploading}
\section{Multimedia}
\section{Internet Access}
\subsection{Basic Authentication}
\subsection{Temporary Sharing}
\section{CDN Protection}
\subsection{Basic Setup}
\subsection{DDNS}
\subsection{ISP Restrictions}
\section{Extra Beautification}
\subsection{Better Index}


\end{document}
